\documentclass[dvipdfmx]{jsarticle}

\usepackage{hyperref}

\usepackage{natbib}
\usepackage[utf8]{inputenc} 
\renewcommand{\refname}{引用文献}
\setlength{\bibhang}{2em} % 同じ文献2行目以降のインデントの大きさ

\begin{document}
\title{心理学的な引用の仕方}
\author{}

\maketitle



例えば文中の引用であれば,\citet{Fukaya2011a}のような形にし,文末の引用であれば次のように書くことになる \citep{Hayashi2015}。

2人以下のケース \citep{Ishii2016}

3人以上のケース \citep{Ishii2017}

% \nobibliography*

翻訳書など \verb+\citep+ では正しく出ない場合は次のように手動で書く (Whiston, 2013 石川・佐藤・高橋監訳 2018) \nocite{Whiston2013}

\bibliographystyle{jecon_jpa}
\bibliography{sample}
% bibファイルの中身を全てリスト化する
\nocite{*}

\end{document}
